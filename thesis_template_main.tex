\documentclass[11pt, a4paper]{report}
\usepackage[utf8]{inputenc}
% ------------ packages section ------------  
% page set up
\usepackage[a4paper, total={6in, 10in}]{geometry}
% epigraph
\usepackage{epigraph}
% quoting
\usepackage{csquotes}
% numbering
\usepackage[utf8]{inputenc}
% language set up
\usepackage[english, main=portuguese]{babel}
\usepackage{csquotes}
% line spacing
\usepackage{setspace}
% adjust margins
\usepackage{changepage}
% number lines
\usepackage{lineno}
% titles
\usepackage{titlesec}
% colors
\usepackage[dvipsnames]{xcolor}
% sorted lists package
\usepackage{enumerate}
% insert graphics
\usepackage{graphicx}
% wrapp figures in text
\usepackage{wrapfig}
% math package
\usepackage{amsmath}
% math symbols
\usepackage{amssymb}
% appendix
\usepackage[title]{appendix}
% letter symbol
\usepackage[misc]{ifsym}
% frame s
\usepackage{tcolorbox}
% listing
\usepackage{listings}
% referencing
\usepackage{nameref}
% caption package - settings
\usepackage[font={scriptsize,sf}, labelfont=bf]{caption}
% glossaries
\usepackage[toc, acronym]{glossaries}
% nomenclature
\usepackage[intoc, portuguese]{nomencl}
% BibLaTeX
\usepackage[backend=biber, style=bwl-FU]{biblatex}


% ------------ links setup ------------
\usepackage{hyperref}
\hypersetup{
	colorlinks=true,
	linkcolor=blue,
	citecolor=blue,
	filecolor=magenta,      
	urlcolor=cyan,
	pdftitle={Overleaf Example},
	pdfpagemode=FullScreen,
}

% ------------ box setup ------------ 
% simple box
\newtcolorbox[auto counter,number within=section]{simplebox}[2][]{
arc=0mm,
%breakable=true,
%coltitle=white,
%coltext=black,
%colback=lightgray,
%colframe=darkgray,
fontupper=\sffamily,
fonttitle=\sffamily\bfseries,
title=\sffamily Box~\thetcbcounter -- #2,#1
}

% code box
\newtcolorbox[auto counter,number within=section]{algbox}[2][]{
arc=0mm,
%coltitle=white,
%coltext=black,
colback=lightgray,
%colframe=darkgray,
fontupper=\sffamily,
fonttitle=\sffamily\bfseries,
title=\sffamily Algorithm~\thetcbcounter -- #2,#1
}

% ------------ theorems ------------ 
%\theoremstyle{definition}
\newtheorem{problem}{Problem}
\newtheorem{definition}{Definition}[section]
\newtheorem{assumption}{Assumption}[section]

% ------------ page setup ------------ 
%\pagecolor{black}
%\color{white}
% Set formats for each heading level
\titleformat{\chapter}[display]
  {\normalfont\sffamily\huge\bfseries}
  {\chaptertitlename\ \thechapter}{20pt}{\LARGE}
\titleformat*{\section}{\Large\bfseries\sffamily}
\titleformat*{\subsection}{\large\bfseries\sffamily}
\titleformat*{\subsubsection}{\normalsize\bfseries\sffamily}


% ------------ listing setup ------------
\lstdefinestyle{mystyle}{
	backgroundcolor=\color{lightgray},   
	commentstyle=\color{darkgray},
	keywordstyle=\color{purple},
	numberstyle=\tiny\color{darkgray},
	stringstyle=\color{teal},
	basicstyle=\color{black}\ttfamily\footnotesize\linespread{2.5},
	breakatwhitespace=false,         
	breaklines=false,                 
	captionpos=b,                    
	keepspaces=true,                 
	numbers=left,                    
	numbersep=2pt,                  
	showspaces=false,                
	showstringspaces=false,
	showtabs=false,                  
	tabsize=3
}
\lstset{style=mystyle}

% ------ set metadata ------ 
\hypersetup{
pdftitle={my_title},
pdfsubject={my_subject},
pdfauthor={my_name},
pdfkeywords={keywords}
}

% ------------ figure setup ------------
% set the figure folder
\graphicspath{{./figs}}
\captionsetup[figure]{labelformat=simple, labelsep=quad}

% ------------ references setup ------------
\addbibresource{refs.bib}

% ------------ names setup ------------
\newcommand{\nameChap}{Capítulo}
\newcommand{\nameSec}{Seção}
\newcommand{\nameAppendix}{Apêndice}
\newcommand{\nameTable}{Tabela}
\newcommand{\nameFigure}{Figura}
\newcommand{\nameEq}{Equação}
\newcommand{\nameBox}{Quadro}
\newcommand{\nameCode}{Algoritmo}

% ------------ glossary setup ------------
\makeglossaries

% define terms
\newglossaryentry{latex}
{
    name=\textbf{latex},
    description={Is a markup language specially suited 
    for scientific documents. Lorem ipsum dolor sit amet consectetur adipiscing elit. Sed ac bibendum orci. Cras erat elit, consequat vel erat ac, tincidunt pulvinar lacus. Pellentesque vitae consectetur quam. Interdum et malesuada fames ac ante ipsum primis in faucibus.}
}
\newglossaryentry{maths}
{
    name=\textbf{mathematics},
    description={Mathematics is what mathematicians do. Lorem ipsum dolor sit amet consectetur adipiscing elit. Sed ac bibendum orci. Cras erat elit, consequat vel erat ac, tincidunt pulvinar lacus. Pellentesque vitae consectetur quam. Interdum et malesuada fames ac ante ipsum primis in faucibus.}
}
\newglossaryentry{amizade}
{
    name=\textbf{amizade},
    description={Is a markup language specially suited 
    for scientific documents. Lorem ipsum dolor sit amet consectetur adipiscing elit. Sed ac bibendum orci. Cras erat elit, consequat vel erat ac, tincidunt pulvinar lacus. Pellentesque vitae consectetur quam. Interdum et malesuada fames ac ante ipsum primis in faucibus.}
}

% define acronyms
\newacronym{gcd}{GCD}{Greatest Common Divisor}
\newacronym{lcm}{LCM}{Least Common Multiple}

% ------------ nomenclature setup ------------
\makenomenclature


\usepackage{subfiles} % Best loaded last in the preamble

% ------------ main document ------------ 
\begin{document}

% first sections
\pagenumbering{roman}
% spacing in the first pages
\singlespacing

% first sections
\section*{Title}
\par Lorem ipsum dolor sit amet consectetur adipiscing elit. Sed ac bibendum orci. Cras erat elit, consequat vel erat ac, tincidunt pulvinar lacus. Pellentesque vitae consectetur quam. Interdum et malesuada fames ac ante ipsum primis in faucibus.
\clearpage	

\section*{Contra-Capa}
\par Lorem ipsum dolor sit amet consectetur adipiscing elit. Sed ac bibendum orci. Cras erat elit, consequat vel erat ac, tincidunt pulvinar lacus. Pellentesque vitae consectetur quam. Interdum et malesuada fames ac ante ipsum primis in faucibus.
\clearpage

\section*{Ficha catalográfica}
\par Lorem ipsum dolor sit amet consectetur adipiscing elit. Sed ac bibendum orci. Cras erat elit, consequat vel erat ac, tincidunt pulvinar lacus. Pellentesque vitae consectetur quam. Interdum et malesuada fames ac ante ipsum primis in faucibus.
\clearpage

\section*{Assinaturas da banca examinadora}
\par Lorem ipsum dolor sit amet consectetur adipiscing elit. Sed ac bibendum orci. Cras erat elit, consequat vel erat ac, tincidunt pulvinar lacus. Pellentesque vitae consectetur quam. Interdum et malesuada fames ac ante ipsum primis in faucibus.
\clearpage

\section*{Agradecimentos}
\par Lorem ipsum dolor sit amet consectetur adipiscing elit. Sed ac bibendum orci. Cras erat elit, consequat vel erat ac, tincidunt pulvinar lacus. Pellentesque vitae consectetur quam. Interdum et malesuada fames ac ante ipsum primis in faucibus.
\clearpage

% table of contents
\tableofcontents
\clearpage	
	
% list of figures
\listoffigures
\clearpage

% list of tables
\listoftables
\clearpage

% list of acron
\printglossary[type=\acronymtype]
\clearpage

% list of symbols
\nomenclature{\(c\)}{Speed of light in a vacuum}
\nomenclature{\(h\)}{Planck constant}
\printnomenclature
\clearpage

% ---------------- Epigraph ---------------- 
\subfile{epigraph}

% ---------------- Chapters setup ---------------- 
\linenumbers % start line numeration
%\doublespacing % line spacing	
\pagenumbering{arabic} % start arabic numeration

% ---------------- Chapters ---------------- 
\subfile{chap_boring}

\subfile{chap_interesting}

\clearpage
%%%% \singlespacing
% ---------------- Glossary ---------------- 
\section*{delete this section}
\par Lorem ipsum dolor sit amet consectetur adipiscing elit. Sed ac bibendum orci. Cras \cite{Popper1934} erat elit, consequat vel erat ac, tincidunt pulvinar lacus. Pellentesque vitae consectetur quam. Interdum et malesuada fames \gls{amizade} ac ante ipsum primis in faucibus. 
The \Gls{latex} typesetting markup language is specially suitable 
for documents that include \gls{maths}. 

Given a set of numbers, there are elementary methods to compute 
its \acrlong{gcd}, which is abbreviated \acrshort{gcd}. This 
process is similar to that used for the \acrfull{lcm}.

\clearpage

\printglossary[title=Glossário, toctitle=Glossário]

\clearpage
% ---------------- References ---------------- 
\printbibliography

% ---------------- Appendix ---------------- 
\begin{appendices}

\chapter{First}
\par Lorem ipsum dolor sit amet consectetur adipiscing elit. Sed ac bibendum orci. Cras erat elit, consequat vel erat ac, tincidunt pulvinar lacus. Pellentesque vitae consectetur quam. Interdum et malesuada fames ac ante ipsum primis in faucibus.
\clearpage

\chapter{Second}
\par Lorem ipsum dolor sit amet consectetur adipiscing elit. Sed ac bibendum orci. Cras erat elit, consequat vel erat ac, tincidunt pulvinar lacus. Pellentesque vitae consectetur quam. Interdum et malesuada fames ac ante ipsum primis in faucibus.
\clearpage

\end{appendices}

\end{document}
